\documentclass{article}
\usepackage{fancyhdr}
\usepackage[graphicx]{realboxes}
\usepackage[margin=1in]{geometry}
\usepackage[Glenn]{fncychap}
\usepackage{listings}
\usepackage[toc]{multitoc}
\usepackage{color}
\usepackage{ctex}
\usepackage{verbatim}

\title{\fontsize{64.0pt}{\baselineskip}\selectfont Let it Rot}
\date{}

\definecolor{dkgreen}{rgb}{0,0.6,0}
\definecolor{gray}{rgb}{0.5,0.5,0.5}
\definecolor{mauve}{rgb}{0.58,0,0.82}


\lstset{frame=tb,
  language=c++,
  aboveskip=3mm,
  belowskip=3mm,
  showstringspaces=false,
  columns=flexible,
  basicstyle={\small\ttfamily},
  numbers=none,
  numberstyle=\tiny\color{gray},
  keywordstyle=\color{blue},
  commentstyle=\color{dkgreen},
  stringstyle=\color{mauve},
  breaklines=true,
  breakatwhitespace=true
  tabsize=4
}


\begin{document}

\begin{titlepage}
\maketitle

\center \includegraphics[scale=0.60]{pics/letitrot.jpg}

\thispagestyle{empty}
\pagebreak
\thispagestyle{empty}

\tableofcontents
\newpage
\end{titlepage}
	
\section{图论}
\pagestyle{fancy}
    \lhead{\large 北京大学-摆烂人}
    \chead{\Huge \thepage}
    \rhead{\Huge \thepage}
    \lfoot{\large PKU-Let it Rot}
    \cfoot{\Huge \thepage}
    \rfoot{\Huge \thepage}

\subsection{欧拉回路} % !!! pb
\lstinputlisting{graph/eulertour.cpp}
\newpage

\subsection{二分图匹配 | 最小边覆盖}
\lstinputlisting{graph/bi_match.cpp}
\newpage

\subsection{网络最大流 | dinic}
\lstinputlisting{graph/dinic.cpp}
\newpage

\subsection{最小费用流 | dijkstra}
\lstinputlisting{graph/dijk-MCMF.cpp}
\newpage

\subsection{最小费用流 | spfa}
\lstinputlisting{graph/spfa-MCMF.cpp}
\newpage

\subsection{二分图最大权匹配} % !!! pb
\lstinputlisting{graph/KM.cpp}
\newpage

\subsection{最小树形图}
抄罗大的,返回值是边的集合,如果没有最小树形图会返回 $-1$,可以修改建图。
\lstinputlisting{graph/D-MST.cpp}
\newpage

\subsection{缩点 | kasaraju}
时间复杂度 $O(\frac{n^2}{w})$, 可以对于边修改不多的图快速计算。
\lstinputlisting{graph/kasaraju.cpp}
\newpage

\subsection{缩点 | Tarjan} % !!! pb
\lstinputlisting{graph/tarjan.cpp}

\subsection{缩点 | 点双} % !!! pb
\lstinputlisting{graph/tarjan_vertex.cpp}

\subsection{缩点 | 边双} % !!! pb
\lstinputlisting{graph/tarjan_edge.cpp}

\subsection{2-Sat}
\lstinputlisting{graph/2-sat.cpp}
\newpage


\section{Math}	

\subsection{万能欧几里得}
\lstinputlisting{math/Euclid.cpp}
\newpage

\subsection{直线下点数|欧几里得}

$n < 2 ^ {32}, 1 \leq m < 2 ^ {32}$

$$
result=\sum_{i=0}^{n-1} \lfloor \frac{ai + b}{m} \rfloor \pmod{2^{64}}
$$

\lstinputlisting{math/floor_sum.cpp}
\newpage

\subsection{扩展欧几里得}
\lstinputlisting{math/exgcd.cpp}

\subsection{扩展中国剩余定理}
\lstinputlisting{math/exCRT.cpp}
\newpage


\subsection{Miller-Rabin}
\lstinputlisting{math/Miller_Rabin.cpp}
\newpage

\subsection{Pollard-rho}
\lstinputlisting{math/pollard_rho.cpp}
\newpage

\subsection{Fast Fourier Transform}
\lstinputlisting{math/FFT.cpp}
\newpage

\subsection{Number Theoretic Transform}
\lstinputlisting{math/NTT.cpp}
\newpage

\subsection{Generating function}
\lstinputlisting{math/GF.cpp}
\newpage

\section{字符串}
\subsection{后缀自动机 | SAM}
需要两倍点数量。
\lstinputlisting{string/SAM.cpp}
\newpage

\subsection{AC 自动机}
\lstinputlisting{string/AC自动机.cpp}
\newpage

\subsection{回文自动机 | PAM}
\lstinputlisting{string/PAM.cpp}
\newpage

\subsection{回文自动机 border 处理}
\lstinputlisting{string/exPAM.cpp}
\newpage

\subsection{后缀数组 | SA}
\lstinputlisting{string/SA.cpp}
\newpage

\subsection{exkmp | Zalgo}
\lstinputlisting{string/Zalgo.cpp}

\subsection{manacher}
\lstinputlisting{string/manacher.cpp}
\newpage

\section{数据结构}
\subsection{区间加区间求和 树状数组}
\lstinputlisting{ds/ex-bit.cpp}
\newpage

\subsection{zkw 线段树}
\lstinputlisting{ds/zkw-sgt.cpp}
\newpage

\subsection{Link Cut Tree}
\lstinputlisting{ds/LCT.cpp}
\newpage

\subsection{FHQ Treap} % !!! pb
\lstinputlisting{ds/fhq.cpp}
\newpage

\section{计算几何}
\subsection{向量}
\lstinputlisting{计算几何/向量.cpp}
\newpage

\subsection{直线半平面}
\lstinputlisting{计算几何/直线半平面.cpp}
\newpage

\subsection{半平面交}
\lstinputlisting{计算几何/半平面交.cpp}
\newpage

\subsection{线段}
\lstinputlisting{计算几何/线段.cpp}
\newpage

\subsection{多边形}
\lstinputlisting{计算几何/多边形.cpp}
\newpage

\subsection{Graham 凸包}
没有判断左下角重点。
结果为逆时针。

\lstinputlisting{计算几何/凸包.cpp}
\newpage
\subsection{上凸壳}
结果显然为顺时针。
\lstinputlisting{计算几何/上凸壳.cpp}
\newpage



\section{不知道叫啥}
\subsection{Pragma}
\lstinputlisting{optimize_tricks/pragma.cpp}

\subsection{Barrett}
\lstinputlisting{optimize_tricks/Barrett.cpp}

\subsection{LCS}
\lstinputlisting{optimize_tricks/lcs.cpp}
\newpage

\subsection{日期公式}
\lstinputlisting{optimize_tricks/zeller.cpp}

\subsection{Xorshift}
\lstinputlisting{optimize_tricks/xorshift.cpp}
\newpage



\section{配置}

\subsection{vimrc}
\lstinputlisting{config/.vimrc}
\subsection{对拍}
需要 chmod +x
\lstinputlisting{config/1.sh}
\subsection{bash 命令}
\large \textbf{ulimit -s 1048576}

\large \textbf{ulimit -m 1048576}
\subsection{随机素数}
920028467

956083511

939822971

947440733

920299999

957673487

963724301



\end{document}
